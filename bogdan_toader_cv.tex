%%%%%%%%%%%%%%%%%%%%%%%%%%%%%%%%%%%%%%%%%
% "ModernCV" CV and Cover Letter
% LaTeX Template
% Version 1.2 (25/3/16)
%
% This template has been downloaded from:
% http://www.LaTeXTemplates.com
%
% Original author:
% Xavier Danaux (xdanaux@gmail.com) with modifications by:
% Vel (vel@latextemplates.com)
%
% License:
% CC BY-NC-SA 3.0 (http://creativecommons.org/licenses/by-nc-sa/3.0/)
%
% Important note:
% This template requires the moderncv.cls and .sty files to be in the same 
% directory as this .tex file. These files provide the resume style and themes 
% used for structuring the document.
%
%%%%%%%%%%%%%%%%%%%%%%%%%%%%%%%%%%%%%%%%%

%----------------------------------------------------------------------------------------
%	PACKAGES AND OTHER DOCUMENT CONFIGURATIONS
%----------------------------------------------------------------------------------------

\documentclass[11pt,a4paper,roman]{moderncv} % Font sizes: 10, 11, or 12; paper sizes: a4paper, letterpaper, a5paper, legalpaper, executivepaper or landscape; font families: sans or roman

\moderncvstyle{classic} % CV theme - options include: 'casual' (default), 'classic', 'oldstyle' and 'banking'
\moderncvcolor{black} % CV color - options include: 'blue' (default), 'orange', 'green', 'red', 'purple', 'grey' and 'black'

\usepackage{lipsum} % Used for inserting dummy 'Lorem ipsum' text into the template

\usepackage[scale=0.85]{geometry} % Reduce document margins 0.75 default
\setlength{\hintscolumnwidth}{3cm} % Uncomment to change the width of the dates column
%\setlength{\makecvtitlenamewidth}{10cm} % For the 'classic' style, uncomment to adjust the width of the space allocated to your name

%----------------------------------------------------------------------------------------
%	NAME AND CONTACT INFORMATION SECTION
%----------------------------------------------------------------------------------------

\firstname{Bogdan} % Your first name
\familyname{Toader} % Your last name

% All information in this block is optional, comment out any lines you don't need
% \title{Curriculum Vitae}
\address{St Anne's College, 56 Woodstock Road}{Oxford, OX2 6HS, United Kingdom}

\mobile{(+44) 07428072233}
%\phone{(000) 111 1112}
%\fax{(000) 111 1113}
\email{toader@maths.ox.ac.uk}
%\homepage{staff.org.edu/~jsmith}{staff.org.edu/$\sim$jsmith} % The first argument is the url for the clickable link, the second argument is the url displayed in the template - this allows special characters to be displayed such as the tilde in this example
\extrainfo{https://people.maths.ox.ac.uk/toader/}
%\photo[70pt][0.4pt]{pictures/picture} % The first bracket is the picture height, the second is the thickness of the frame around the picture (0pt for no frame)
%\quote{"A witty and playful quotation" - John Smith}

%----------------------------------------------------------------------------------------

\begin{document}

%----------------------------------------------------------------------------------------
%	COVER LETTER
%----------------------------------------------------------------------------------------

% To remove the cover letter, comment out this entire block

%\clearpage
%
%\recipient{HR Department}{Corporation\\123 Pleasant Lane\\12345 City, State} % Letter recipient
%\date{\today} % Letter date
%\opening{Dear Sir or Madam,} % Opening greeting
%\closing{Sincerely yours,} % Closing phrase
%\enclosure[Attached]{curriculum vit\ae{}} % List of enclosed documents
%
%\makelettertitle % Print letter title
%
%\lipsum[1-3] % Dummy text
%
%\makeletterclosing % Print letter signature
%
%\newpage

%----------------------------------------------------------------------------------------
%	CURRICULUM VITAE
%----------------------------------------------------------------------------------------

\makecvtitle % Print the CV title
% \vspace{-3em}



%----------------------------------------------------------------------------------------
%	RESEARCH INTERESTS SECTION
%----------------------------------------------------------------------------------------

\section{Research Interests}
My research is focused on grid-free compressed sensing, or super-resolution. 
So far I have developed theory for the stability of the super-resolution problem 
with non-negative measures. 
Current and future work include extending the theory to the recovery 
of curves in two dimensions, working on algorithms for super-resolution and
applying them to large scale data from scientific experiments.
\newline

More generally, I am interested in the broader fields of compressed sensing, 
mathematical signal processing, optimisation, machine learning 
and their application to problems arising in other scientific fields.



%----------------------------------------------------------------------------------------
%	RESEARCH SUMMARY SECTION
%----------------------------------------------------------------------------------------

%\section{Research Summary}
%My research is motivated by the need for fast and robust reservoir simulators for the oil and gas industry. I focus on the decoupling strategy used for the preconditioning in reservoir simulation.

%----------------------------------------------------------------------------------------
%	EDUCATION SECTION
%----------------------------------------------------------------------------------------

\section{Education}

\cventry{Oct 2018--present}{Enrichment Student}
        {Alan Turing Institute}{London, UK}{}
        {
          Six months placement at UK's national institute for
          data science and artificial intelligence.
        }

\cventry{2015--present}{PhD candidate in Mathematics}
        {University of Oxford}{Oxford, UK}{}
        {
          Industrially Focused Mathematical Modelling 
          (EPSRC Centre for Doctoral Training) 
          in collaboration with the National Physical Laboratory (NPL). 
          \\
          Main research project on \textbf{theory and algorithms for super-resolution.}
          \\ 
          Expected end date: August 2019.
        }
  %\smallitem[0em]{\textit{Thesis title}}{Source reconstruction from hydrophone data}%
  \smallitem{\textit{Advisors}}{Prof Jared Tanner, Dr Andrew Thompson}%
  \smallitem{\textit{Industrial supervisors}}{Dr Stephane Chretien, Dr Peter Harris (NPL)}

\cventry{2009--2013}{BSc (Hons) Computer Science and Mathematics with 
        Industrial Experience}
        {University of Manchester}{Manchester, UK}{}
        {
          First class degree with final grade above 80\%.
        }
  
\cventry{2005--2009}{Romanian Baccalaureate}
        {"Gheorghe Munteanu Murgoci" National College}
        {Braila, Romania}{}
        {Final grade of 9.91 (out of a maximum of 10).}


%----------------------------------------------------------------------------------------
%	PUBLICATIONS
%----------------------------------------------------------------------------------------
%\newpage
\section{Publications}

%\textbf{Articles submitted to peer-reviewed journals}

%TODO: move the items to the right (including the numbers)
\begin{enumerate}

  \item
\textbf{Non-negative super-resolution is stable} \\
Armin Eftekhari, Jared Tanner, Andrew Thompson, Bogdan Toader and Hemant Tyagi\\
{\em 2018 IEEE Data Science Workshop, DSW 2018 - Proceedings page 100-104},
2018 
\vspace{1em}
  
  \item
\textbf{Sparse non-negative super-resolution -- simplified and stabilised} \\
Armin Eftekhari, Jared Tanner, Andrew Thompson, Bogdan Toader and Hemant Tyagi\\
Submitted to {\em Applied and Computational Harmonic Analysis},
2018 \\
Preprint on arXiv: https://arxiv.org/abs/1804.01490
\vspace{1em}

  \item
\textbf{The dual approach to non-negative super-resolution: 
  impact on primal\\ reconstruction accuracy}\\
Stephane Chretien, Andrew Thompson, Bogdan Toader\\
Submitted to {\em SampTA 2019},\\
Preprint on arXiv: https://arxiv.org/abs/1904.01926 
\vspace{1em}
\newpage
  \item
\textbf{A perturbation analysis of the super-resolution problem
in the presence of noise}\\
Stephane Chretien, Andrew Thompson, Bogdan Toader\\
{\em Work in progress}
\vspace{1em}

  \item
\textbf{An improved level method for super-resolution}\\
Stephane Chretien, Andrew Thompson, Bogdan Toader\\
{\em Work in progress}
\vspace{1em}

\end{enumerate}
% Add subsections later when there's more stuff in each

%\textbf{Articles in peer-reviewed conference proceedings}

%\textbf{Work in progress}

%\textbf{Technical Reports}


        

%----------------------------------------------------------------------------------------
%	RESEARCH EXPERIENCE SECTION
%----------------------------------------------------------------------------------------

\section{Other Research Experience}


\cventry{Jul--Sep 2016}{Deflating Magnetic Oscillations}
        {Culham Centre for Fusion Energy}{Abingdon, UK}{}{
          Used deflation to find multiple periodic solutions to a 
          system of ODEs that describes the behaviour of plasma.
          In collaboration with Culham Centre for Fusion Energy.
        }

        \smallitem{\textit{Supervisors}}{
          Prof Patrick Farrell (Oxford), 
          Dr Wayne Arter (CCFE)
        }

\cventry{May--Jul 2016}
        {Improved Source Reconstruction from Hydrophone Data}
        {National Physical Laboratory}{London, UK}{}{
          Analysed how compressed sensing 
          can be applied to a problem on ship localisation
          from measurements of the sound in the shipping lane, 
          proposed by the National Physical Laboratory. An extension of this work 
          to grid-free compressed sensing applied to the same problem 
          has been the focus of my PhD project for the following three years.
        }

        \smallitem{\textit{Supervisors}}{
          Prof Jared Tanner (Oxford), Dr. Andrew Thompson (Oxford),
          Dr Peter Harris (NPL), Dr Stephane Chretien (NPL)
        }

\cventry{2012--2013}
        {Formal Verification of Dynamical System}
        {University of Manchester}{UK}{}{
          Final year undergraduate thesis on using the automatic theorem 
          prover MetiTarski to analyse equilibrium and stability properties 
          of dynamical systems.
        }

        \smallitem{\textit{Supervisor}}{
          Dr Eva Navarro-L\'opez
        }


% \section{Conferences and Seminars}
\section{Presentations}

\cvitem{Feb 2019}{
        \textit{InFoMM Group Meeting},
        Oxford, UK -- oral presentation
}

\cvitem{Jul 2018}{
        \textit{Curves and Surfaces Conference},
        Arcachon, France -- oral presentation
}

\cvitem{Jun 2018}{
        \textit{6th IMA Conference on Numerical Linear Algebra and Optimization},
        Birmingham, UK -- oral presentation 
}

\cvitem{Jun 2018}{
        \textit{2018 IEEE Data Science Workshop}, 
        Lausanne, Switzerland -- poster presentation
      }

\cvitem{Mar 2018}{
        \textit{InFoMM Annual Meeting 2018}, 
        Oxford, UK -- oral presentation
      }

\cvitem{Mar 2018}{
        \textit{Numerical Analysis Seminar}, 
        Oxford, UK -- oral presentation
      }

\cvitem{Feb 2018}{
        \textit{Research Workshop on Optimization and Big Data},
        KAUST, Saudi Arabia -- poster presentation
      }

\cvitem{Jan 2018}{
        \textit{SIAM UKIE Annual Meeting}, 
        Southampton, UK -- poster presentation
      }
       
\cvitem{May 2017}{
        \textit{InFoMM Group Meeting}, 
        Oxford, UK -- oral presentation 
      }

\cvitem{Mar 2017}{ 
        \textit{InFoMM Annual Meeting 2017}, 
        Oxford, UK -- poster presentation
      }


%----------------------------------------------------------------------------------------
\section{Other Academic Events}

\cvitem{Jan 2019}{
    \textit{Mathematics of Imaging CIRM Winter School}, 
    Marseille, France
}
\cvitem{Jun 2018}{
    \textit{142nd European Study Group with Industry},
    Palanga, Lithuania  \newline
    Worked on predicting the sustainable income of loan 
    applicants according to rules from the central bank.
}
\cvitem{Jun 2017}{
    \textit{Summer School on Structured Regularization 
      for High-Dimensional Data Analysis},
    \newline Henri Poincare Institute, Paris, France
}
\cvitem{Dec 2016}{
    \textit{Data Study Group}, 
    Alan Turing Institute, London, UK \newline
    Implemented network model to solve an air traffic prediction
    problem proposed by Airbus.
}
\cvitem{May 2016}{
    \textit{116th European Study Group With Industry}, 
    Durham, UK \newline
    Implemented mixed integer programming solution to problem 
    on scheduling field trials proposed by Syngenta.
}
\cvitem{Mar 2016}{
    \textit{InFoMM Graduate Modelling Camp}, 
    Oxford, UK \newline
    Worked on calculating trajectory of footballs. 
    Won IMA Best Team Performance prize.
}


%----------------------------------------------------------------------------------------
%	AWARDS SECTION
%----------------------------------------------------------------------------------------

\section{Awards}


\cvitem{2018}{
    \textit{Travel Award},
    SIAM UKIE Annual Meeting,
    Southampton, UK 
}
\cvitem{2016}{
    \textit{IMA Best Team Performance Prize},
    InFoMM Graduate Modelling Camp,
    Oxford, UK
}
\cvitem{2015}{
    \textit{EPSRC InFoMM CDT Studentship},
    Oxford, UK
}
\cvitem{2010}{
    \textit{Golden Anniversary Prize},
    University of Manchester, UK \newline
    For excellence in first year studies, awarded to the 
    students with the first five highest grades in the first year.
}
\cvitem{2006--2009}{
    \textit{Bronze Medal at the National Mathematical Olympiad},
    Romania in 2007, 2008 and 2009 \newline
    Won various prizes at other national and regional mathematics contests, 
    including the national contest organised by the editors of the 
    Romanian mathematical journal "Gazeta Matematica" for students who 
    regularly send solutions to the problems published in the journal.
}


\section{Teaching Experience}


\cventry{2016--2018}{Continuous Optimisation}
        {University of Oxford}{Oxford, UK}{}
        {
          Teaching assistant for 4th year undergraduate course.
          Responsible with marking weekly assignments and solving
          problems on the board during classes, to groups
          of up to 15 students.
          Ran revision classes independently to groups of 30 students.
        }
        \smallitem{Lecturer}{Prof Coralia Cartis}

\cventry{2016--2018}{Integer Programming}
        {University of Oxford}{Oxford, UK}{}
        {
          Teaching assistant for 3rd year undergraduate course.
          Responsible with marking weekly assignments and solving
          problems on the board during classes, to groups
          of up to 15 students.
          Ran revision classes independently to groups of 30 students.
        }
        \smallitem{Lecturer}{Prof Raphael Hauser}

\cventry{2016--2017}{Computational Mathematics}
        {University of Oxford}{Oxford, UK}{}
        {
          Lab demonstrator for 1st year undergraduate Matlab classes.
          Presented new material in the form of live computer demo to groups
          of up to 20 students.
        }
        \smallitem{Lecturer}{Dr Andrew Thompson}

\cventry{2010--2011}{Peer Assisted Study Sessions Leader}
        {University of Manchester}{Manchester, UK}{}
        {
          Weekly sessions with 1st year 
          undergraduate students, in groups of around 6,
          assisting them with basic mathematics and programming.
        }

%------------------------------------------------
%----------------------------------------------------------------------------------------
%	INDUSTRY EXPERIENCE
%----------------------------------------------------------------------------------------

\section{Industry Experience}

\cventry{2013--2015}{Technology Associate}
        {Morgan Stanley}{London, UK}{}
        {
          Worked in the Pricing Technology team for the Interest 
          Rate Derivatives business, where I used Scala and Perl 
          to build and improve the pricing tools used 
          by quantitative analysts and traders. 
        }

\cventry{Jun--Jul 2012}{Summer Intership, Technology}
        {Credit Suisse}{London, UK}{}
        {
          Part of the team that maintains the Unix servers in the 
          EMEA region. On top of handling daily requests from users, 
          I improved my shell scripting skills.
        }

\cventry{2011-2012}{Industrial Placement, Technology}
        {Morgan Stanley}{London, UK}{}
        {
          Worked as part of the Institutional Securities Group Technology 
          division in one of the teams developing and supporting the equities 
          trading systems. Acquired experience of working with large sets of data.
        }

\section{Outreach}

\cventry{2017--2018}{Lord Williams's School}
        {Thame, Oxfordshire, UK}{}{}
        {
          Ran two outreach sessions (in 2017 and in 2018 respectively) 
          aimed at pre-final year 
          students about doing research in applied mathematics.
          Presented material on the mathematics of machine learning.
        }

%----------------------------------------------------------------------------------------
%	COMPUTER SKILLS SECTION
%----------------------------------------------------------------------------------------

\section{Computer Skills}

\cvitem{Advanced}{\textsc{Matlab}, \textsc{Python}, 
          \textsc{Java}, \textsc{Scala},
          \textsc{Linux}, \LaTeX}
\cvitem{Intermediate}{\textsc{C}, \textsc{C++}, \textsc{Perl}} 

\section{Other}

\cvitem{Languages}{Romanian (native), English (fluent)}{}{}{}

\cvitem{Hobbies}{Gliding, running, hiking, climbing, skiing, guitar}{}{}{}

%\section{References}
%
%\begin{enumerate}
%
%  \item
%\textbf{Prof Jared Tanner} \\
%University of Oxford, Numerical Analysis Group\\
%tanner@maths.ox.ac.uk
%\vspace{1em}
%
%  \item
%\textbf{Dr Stephane Chretien} \\
%National Physical Laboratory, Data Science Department\\
%stephane.chretien@npl.co.uk
%
%\end{enumerate}






\end{document}
